
In this proof we assumed that the random string $r$ comes from a very specific distribution
that corresponds to cycle matrices.
Now we need to show that the general problem (where $r$ comes from a
random uniform distribution of $\ell$ bits) can be reduced into this previous scenario.

We proceed as follows.
Let the length of the random string be
$\ell=\left\lceil 3\cdot \log_2 n\right\rceil \cdot n^4$.
We view the random string $r$ as $n^4$ blocks of $\left\lceil 3\cdot \log_2 n\right\rceil$
bits and we generate a random string $r'$ of length $n^4$ such that each bit in $r'$
is 1 if and only if all the bits in the corresponding block of $r$ are equal to 1.
This way, the probability that the $i$-th bit of $r'$ equals 1 is $\Pr[r'_i=1]\approx\frac{1}{n^3}$ for every $i$.

Then we create an $n^2\times n^2$ matrix $M$ whose entries are given by the bits of $r'$.
Let $x$ be the number of one entries in the matrix $M$.
The expected value for $x$ is $\frac{n^4}{n^3}=n$.
And the probability that $x$ is exactly $n$ is noticeable. To prove that, we can use
Chebyshev's inequality:
$$\Pr[|x-n|\geq n]\leq\frac{\sigma^2}{n^2}=
\frac{n^4\cdot \frac{1}{n^3}\cdot\left(1-\frac{1}{n^3}\right)}{n^2}<\frac{1}{n}.$$
So we have $\Pr[1\leq x\leq 2n-1]>\frac{n-1}{n}$.
And the probability $\Pr[x=k]$ is maximal for $k=n$, so we conclude that
$\Pr[x=n]>\frac{n-1}{n(2n-1)}>\frac{1}{3n}$.

Now suppose that this event ($x=n$) occurred and we have exactly $n$ entries equal to 1
in matrix $M$. What is the probability that those $n$ entries are all in different rows
and are all in different columns?

We can think about the problem this way: after $k$ one entries have been added to the matrix,
the probability that a new entry will be in a different row and different column is given by
$\left(1-\frac{k}{n^2}\right)^2$. Multiplying all these values we get

\begin{align*}
\Pr[\text{no collision}] &\geq \left(1-\frac{1}{n^2}\right)^2 \cdot \left(1-\frac{2}{n^2}\right)^2
\cdots \left(1-\frac{n-1}{n^2}\right)^2 \\
& > 1 - 2\left(\frac{1}{n^2} + \frac{2}{n^2} +\cdots + \frac{n-1}{n^2}\right)
= 1 - \frac{n-1}{n} = \frac{1}{n}.
\end{align*}

Now assume that this event happened: the matrix $M$ has exactly $n$ entries equal to 1
and they are all in different rows and different columns.
Then we can define a new $n\times n$ matrix $M_c$ by selecting only those $n$ rows
and $n$ columns of $M$. By construction, $M_c$ is a permutation matrix.
The probability that $M_c$ is a cycle matrix is $\frac{(n-1)!}{n!}=\frac{1}{n}$.
An example is shown in Figures~\ref{fig:n2}~and~\ref{fig:n}.

\begin{figure}[ht]
	\centering
		\includegraphics[height=8cm]{Old Scribe Notes/n2.png}
	\caption{Matrix $M$ which is $n^2\times n^2$ for $n=8$.}
	\label{fig:n2}
\end{figure}

\begin{figure}[ht]
	\centering
		\includegraphics[height=4cm]{Old Scribe Notes/n.png}
	\caption{Matrix $M_c$ which is $n\times n$ for $n=8$. The construction worked,
	         because $M_c$ is a cycle matrix.}
	\label{fig:n}
\end{figure}


Now let's join all those probabilities. The probability that $M_c$ is a cycle matrix is at least
$$\frac{1}{3n}\cdot \frac{1}{n}\cdot \frac{1}{n} > \frac{1}{3n^3}.$$

If we repeat this process $n^4$ times, then the probability that $M_c$ is a cycle matrix in at least one iteration is at least
$$1-\left(1-\frac{1}{3n^3}\right)^{n^4}\approx 1-e^{-\frac{n}{3}} = 1-\mathsf{negl}(n).$$


\bigskip
The proof system works as follows. Given a random string $r$, the prover $P$ tries
to execute the construction above to obtain a cycle matrix.
If the construction fails, the prover simply reveals all the bits in the string $r$
to the verifier, who checks that the constructions indeed fails.
If the construction succeeds, the prover reveals all the entries in the random string $r$
that correspond to values in the matrix $M$ which are not used in matrix $M_c$.
The verifier will check that all these values for matrix $M$ are indeed 0.

Then the prover proceeds as in the previous scenario using matrix $M_c$: he
reveals the transformation $\phi$ and opens all the non-edges.

This process is repeated $n^4$ times. Or, equivalently, a big string of length
$\left\lceil 3\cdot \log_2 n\right\rceil \cdot n^4\cdot n^4$ is used and they are all
executed together. This produces a zero knowledge proof.

\textit{Completeness:} if $P$ knows the Hamiltonian cycle of $G$,
then he will be able to find a suitable transformation $\phi$ whenever a cycle graph is
generated by the construction.

\textit{Soundness:} if $P$ is lying and trying to prove a false statement, then he will
get caught with very high probability. If any of the $n^4$ iterations produces a cycle
graph, then $P$ will be caught. So the probability that he will be caught is
$1-e^{-\frac{n}{3}} = 1-\mathsf{negl}(n)$.

\textit{Zero Knowledge:} again $V$ cannot get any information if the construction succeeds.
And if the construction doesn't succeed, all $V$ gets is the random string $r$, which also
doesn't give any information.
\qed


\begin{theorem}
For any language $L$ in $NP$, there is a non-interactive zero-knowledge (NIZK) proof
in the hidden-bit model (HBM) for the language $L$.
\end{theorem}
\proof
The language $L^*$ of Hamiltonian graphs is $NP$-complete. So any problem in $L$ can
be reduced to a problem in $L^*$. More precisely, there is a polynomial-time function
$f$ such that
$$x\in L \Longleftrightarrow f(x)\in L^*.$$
So given an input $x$, the prover can simply calculate $f(x)$ and
produce a NIZK proof in the hidden-bit model for the fact that $f(x)\in L^*$.
Then the verifier just needs to calculate $f(x)$ and check if the proof for the fact
$f(x)\in L^*$ is correct.
\qed

\begin{theorem}\label{the:NIZK_NP}
For any language $L$ in $NP$, there is a non-interactive zero-knowledge (NIZK) proof
in the common reference string (CRS) model for the language $L$.
\end{theorem}
\proof
In Theorem~\ref{thm:NIZK-amplify} it was shown that any NIZK proof in the hidden-bit model can
be converted into a NIZK proof in the standard (common reference string) model by using
a trapdoor permutation.
\qed
