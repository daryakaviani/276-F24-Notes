% Add Eval + and x, noting the noise growth of each 
\begin{itemize}
    \item $\mathsf{Dec}(s, C)$: Compute $v = Cs$. If $||v||_\infty < q/4$, i.e. each entry of $v$ is less than $q/4$, output $0$, else output $1$. (Note the exact choice of threshold is somewhat arbitrary.)
    
    \item $\mathsf{Eval}(+, C_1, C_2)$: Output $C = C_1 + C_2$. 

    \begin{itemize}
        \item Notice that $Cs = C_1s + C_2s = (m_1 Gs + e_1) + (m_2 G + e_2) = (m_1 + m_2) Gs + (e_1 + e_2)$. 

        \item Thus if the original ciphertexts errors $||e_1||_{\infty}, ||e_2||_{\infty} \leq B$, then the new cipertext error is bounded by $2B$. 
    \end{itemize} 

    \item $\mathsf{Eval}(+, C_1, C_2)$: Output $C = \mathsf{Flatten}(C_1) \times C_2$. 

    \begin{itemize}
        \item We utilize the $\mathsf{Flatten}$ operation so that the dimensions match for matrix multiplication. 

        \item Notice that 
        \begin{align*}
            Cs & = \mathsf{Flatten}(C_1) \times C_2 s \\
            & = \mathsf{Flatten}(C_1) \times (m_2 G s + e_2) \\ 
            & = m_2 (\mathsf{Flatten}(C_1) G) s + \mathsf{Flatten}(C_1) e_2 \\ 
            & = m_2 (C_1 s) + \mathsf{Flatten}(C_1) e_2 \\
            & = m_2 (m_1 Gs + e_1) + \mathsf{Flatten}(C_1) e_2 \\ 
            & = (m_1 m_2) G s + (m_2 e_1 + \mathsf{Flatten}(C_1) e_2). 
        \end{align*} 

        \item Since each entry of $\mathsf{Flatten}(C_1)$ is in $\{0,1\}$, each entry of $\mathsf{Flatten}(C_1) e_2$ is a subset sum of $e_2 \in \mathbb{Z}_q^{\ell}$. Thus if the original ciphertexts errors $||e_1||_{\infty}, ||e_2||_{\infty} \leq B$, then the new ciphertext error is bounded by $(1+\ell)B$. 
    \end{itemize}
\end{itemize}

% Analyze overall noise growth 

Notice that as $\mathsf{Eval}$ is applied iteratively to ciphertexts in order to implement an arithmetic circuit of depth $d$, the noise will stay bounded by $(\ell+1)^d B$. 
As long as the error stays below the threshold used by $\mathsf{Dec}$, i.e. $(\ell+1)^d B << q/4$, then correctness will hold. 
Recall that $\ell = (n+1) \log(q)$ where $q$ is typically exponential in $n$. 
Then the error will be manageable as long as $d$ is sublinear in $n$, e.g. $d = n^{0.99}$. 

Further, notice that since swapping ordering of $C_1$ vs $C_2$ changes the noise growth of multiplication from $m_2 e_1 + \mathsf{Flatten}(C_1) e_2$ to $m_1 e_2 + \mathsf{Flatten}(C_2) e_1$, we can optimize based on which ciphertext started with more noise than the other. 
This can be leveraged to get polynomial noise growth instead of exponential. 

% Add Eval for NAND 

Now, we have cheated slightly in the above---notice that our construction for $\mathsf{Eval}(+, C_1, C_2)$ implements addition mod $q$, but we actually wanted addition mod 2. 
One way to address this is to simply implement NAND, which on its own is a complete gate set that can implement any circuit: 
\begin{itemize}
    \item $\mathsf{Eval}(\text{NAND}, C_1, C_2)$: Output $C = I - \mathsf{Flatten}(C_1) \times C_2$. 
\end{itemize}

Altogether, we have achieved \emph{leveled} FHE. 

% Present bootstrapping and how it ensures noise doesn't grow too large 

\subsection{Bootstrapping} 

The goal of bootstrapping is to reset the ciphertext's noise back to a lower level after it has built up too much. 
This is done by letting the server re-encrypt the ciphertext \emph{without} using the secret key $s$. 
How is this possible? 
Simply evaluate $\mathsf{Dec}$ homomorphically! 
In particular, let $P_{\mathsf{Dec}, C}$ be a circuit which on input $s$ outputs $m = \mathsf{Dec}(s, C)$. 
Let $s_1, s_2, \ldots, s_{\ell} \in {0,1}$ be $s$ written in binary. 
We will set the evaluation key $ek = (ek_1, \ldots, ek_{\ell})$ for $ek_i = \mathsf{Enc}(s, s_i)$, i.e. it gives us an encrypted copy of the secret key. 
Then whenever we need to reduce the noise of a ciphertext $C$, we can compute $P_{\mathsf{Dec}, C}$ and run $\mathsf{Eval}(P_{\mathsf{Dec}, C}, ek_1 \ldots ek_\ell)$ to get a fresh ciphertext $\hat{C}$. 
Notice that the noise of $\hat{C}$ only depends on the noise of the input ciphertexts $ek_i$, which are fresh, and the depth $d_P$ of the circuit $P_{\mathsf{Dec}, C}$, which is independent of the noise of $C$. 
Thus $\hat{C}$ will have noise bounded by $(\ell + 1)^{d_P} B$. 
As long as our leveled scheme supports circuits of depth $d_p+1$, we can achieve arbitrary depth by bootstrapping after each operation. 
Of course, we'd like to avoid this as much as possible since doing the decryption operation homomorphically is expensive. 

% Circular security issue 

In order for this approach to work, security needs to be maintained even though we're releasing encryptions of the secret key. 
This is known as \emph{circular security}. 
There are no known attacks on circular security for GSW and other commonly used leveled FHE constructions. 
However, they have not been proven to be circularly secure, and there are encryption schemes for which circular security is known to not hold for certain cycle lengths. 
An alternative is using a new secret key to encrypt the old secret key. 

% Construction of CCA-2 secure IBE 

\section{CCA-2 Secure Encryption from IBE} 

Finally, we will see how to utilize CPA secure IBE to construct not only CCA-1 but CCA-2 secure encryption. 
Let $(G, K, E, D)$ be a CPA secure IBE scheme and $(\mathsf{Gen}_{sig}, \mathsf{Sign}, \mathsf{Verify})$ be a digital signature scheme. 
Then construct a CCA-2 secure encryption scheme as follows: 
\begin{itemize}
    \item $\mathsf{KeyGen}(1^n)$: 
    \begin{enumerate}
        \item Compute $(pk, sk) \leftarrow G(1^n)$. 
        
        \item Output $pp = pk$ and $msk = sk$. 
    \end{enumerate}

    \item $\mathsf{Enc}(pk, m)$: 
    \begin{enumerate}
        \item \textbf{Compute $(vk_s, sk_s) \leftarrow Gen_{sig}(1^n)$.}
        
        \item \textbf{Let $id = vk_s$.}

        \item Let $c = E(pp, id, m)$. 

        \item \textbf{Compute $\sigma = \mathsf{Sign}(sk_s, c)$.} 

        \item \textbf{Output $(id, c, \sigma)$.} 
    \end{enumerate}

    \item $\mathsf{Dec}(sk, (id,c,\sigma))$: 
    \begin{enumerate}
        \item \textbf{If $\mathsf{Verify}(id, c, \sigma) = 0$ then abort.} 

        \item Compute $sk_{id} \leftarrow K(sk, id)$. 

        \item Output $D(sk_{id}, c)$. 
    \end{enumerate}
\end{itemize}
The bolded lines are the ones that have been changed compared to the CCA-1 construction. 
The changes restrict you to only generating ciphertexts for $id$'s you sampled yourself. 
Note that for this construction we need slightly stronger forgery protection for the digital signature scheme than we've considered previously. 
Namely, in the forgery security game the attacker is now allowed to output $(m^*, \sigma^*)$ as long as the tuple is fresh, even if $m^*$ on its own is not. 
Notice that this stronger notion holds for any deterministic signature scheme, including the BLS scheme we saw. 
It is also sufficient for our purposes here to utilize a one-time signature scheme. 