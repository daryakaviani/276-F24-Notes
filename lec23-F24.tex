\section*{Exercises}
\begin{exercise}[Leaky ZK proof] Formally define:
\begin{enumerate}
  \item 
What it means for an  interactive proof $(P,V)$ to be \textbf{first-bit leaky} zero-knowledge, where we require that the protocol doesn't leak anything more than the first bit of the witness.

\item What it means for an  interactive proof $(P,V)$ to be \textbf{one-bit leaky} zero-knowledge, where we require that the protocol doesn't leak anything more than one bit that is an arbitrary adversarial chosen function of the witness.
    \end{enumerate}
\end{exercise}

\begin{exercise}[Proving OR of two statements] Give a statistical zero-knowledge proof system $\Pi = (P,V)$ (with efficient prover) for the following language.
    \[ L = \left\{((G_0,G_1),(G_0',G_1'))\left| G_0 \simeq G_1 \bigvee G_0' \simeq G_1'\right.\right\}\]\\
    \textbf{Caution:} Make sure the verifier doesn't learn which of the two pairs of graphs is isomorphic.
\end{exercise}

\begin{exercise} [ZK implies WI] Let $L \in NP$ and let $(P,V)$ be an interactive proof system for $L$. We say that $(P,V)$ is \emph{witness indistinguishable (WI)} if for all PPT $V^*$, for all $x \in L$, distinct witnesses $w_1, w_2 \in R_L(x)$ and  auxiliary input $z\in \binset{*}$, the following two views are computationally indistinguishable:
\[View_{V^*} \left(P(x,w_1) \leftrightarrow V^*(x,z) \right) \simeq_c View_{V^*} \left(P(x,w_2) \leftrightarrow V^*(x,z) \right).\]
\begin{enumerate}
\item Show that if $(P,V)$ is an efficient prover zero-knowledge proof system, then it is also witness indistinguishable.

\item Assume $(P,V)$ is an efficient prover zero-knowledge proof system. We have seen in the exercise that $(P,V)$ is also witness indistinguishable. Define $(\tilde P, \tilde V)$ to repeat $(P,V)$ independently for $k$ times \emph{in parallel} ($k$ is a polynomial), and $\tilde V$ accepts if and only if $V$ accepts in all the executions. Prove that $(\tilde P, \tilde V)$ is still witness indistinguishable.
\end{enumerate}    
\end{exercise}

\begin{exercise}
\textbf{Multi-statement NIZK.} The NIZK proof system we constructed in class required a fresh common random string (CRS) for each statement proved. In various settings we would like to reuse the same random string to prove multiple theorem statements while still preserving the zero-knowledge property.
    
    A multi-statement NIZK proof system $(K,P,V)$ for a language $L$ with corresponding relation $R$ is a NIZK proof system for $L$ with a stronger zero-knowledge property, defined as follows: $\exists$ a PPT machine $\mathcal{S} = (\mathcal{S}_1,\mathcal{S}_2)$ such that $\forall$ PPT machines $A_1$ and $A_2$ we have that:
    \[\left|\Pr\left[\begin{split}\sigma \gets K(1^\kappa),\\ (\{x_i,w_i\}_{i \in [q]},\textsf{state}) \gets A_1(\sigma),\\ \text{ such that } \forall i \in [q], (x_i,w_i)\in R\\\forall i \in [q],  \pi_i \gets P(\sigma, x_i,w_i);\\
    A_2(\textsf{state}, \{\pi_i\}_{i \in [q]}) =1\end{split}\right]
    -
    \Pr\left[\begin{split}(\sigma,\tau) \gets \mathcal{S}_1(1^\kappa),\\ (\{x_i,w_i\}_{i \in [q]},\textsf{state}) \gets A_1(\sigma),\\\text{ such that } \forall i \in [q], (x_i,w_i)\in R\\\forall i \in [q],  \pi_i \gets \mathcal{S}_2(\sigma, x_i,\tau);\\ A_2(\textsf{state}, \{\pi_i\}_{i \in [q]})=1\end{split}\right]\right|
    \leq \textsf{negl}(\kappa).
    \]
    
    Assuming that a single statement NIZK proof system $(K,P,V)$ for NP exists, construct a multi-statement NIZK proof system $(K',P',V')$ for NP.\\
\textbf{Hint:} Let $g: \{0,1\}^\kappa \rightarrow \{0,1\}^{2\kappa}$ be a length doubling PRG. Let $K'$ output the output of $K$ along with $y$, a random $2\kappa$ bit string. To prove $x \in L$ the prover $P'$ proves that $\exists (w,s)$ such that either $(x,w)\in R$ or $y = g(s)$.
\end{exercise}


\paragraph{Answer}
Construction:
\[ K'(1^n): 
\begin{array}{l l} 
\sigma \leftarrow K(1^{\kappa}) \\  
y \leftarrow \{0,1\}^{2\kappa} \\
\text{Output } \sigma' = (\sigma, y) \\
\end{array} \]

\[ P'(\sigma'=(\sigma, y), x, w): 
\begin{array}{l l} 
\pi \leftarrow P(\sigma, (x,y), w)\text{ if }x \in L \text{ or } \exists s\text{ s.t. } g(s)=y \\
\text{Output }\pi \\
\end{array} \]

\[ V'(\sigma', x, \pi): 
\begin{array}{l l} 
\text{Output } V(\sigma, x, \pi)
\end{array} \]

The completeness of this systems follows from the completeness of $(K, P, V)$. The soundness is based on the PRG $g$. Since $y$ is a randomly chosen string, for almost all possible choice of $y$, it is not in the range of $g$, and thus $x \in L$ with overwhelming probability. Therefore, the soundness property also follows from that of $(K,P,V)$

To prove zero-knowledge property, we construct the simulator $\mathcal{S}_1, \mathcal{S}_2$ as follows:
\[ \mathcal{S}_1(1^\kappa): 
\begin{array}{l l} 
\sigma \leftarrow K(1^\kappa) \\
s \leftarrow {0,1}^{n}, y' = g(s) \\
\text{Output } ((\sigma, y'), s)
\end{array}
\text{, }
\mathcal{S}_2(\sigma'=(\sigma, y'), x, \tau=s): 
\begin{array}{l l} 
\text{Output } P(\sigma, (x,y'), s) \\
\end{array} \]

We show that the proofs from $S_2$ are indistinguishable from those from $P'$ via hybrid argument. Consider following sequence of the proofs ($\{\pi_i, i \in [q]\}$).

\[ 
\mathcal{H}_0: 
\begin{array}{l l} 
\sigma \leftarrow K(1^n) \\
y \leftarrow \{0,1\}^{2n} \\
\{\pi_i \leftarrow P(\sigma, (x_i, y), w_i)\}
\end{array} 
\mathcal{H}_1: 
\begin{array}{l l} 
\sigma \leftarrow K(1^n) \\
s \leftarrow \{0,1\}^{n},  y = g(s) \\
\{\pi_i \leftarrow P(\sigma, (x_i, y), w_i)\}
\end{array} 
\mathcal{H}_2: 
\begin{array}{l l} 
\sigma \leftarrow K(1^n) \\
s \leftarrow \{0,1\}^{n},  y = g(s) \\
\pi_1 \leftarrow P(\sigma, (x_i, y), s) \\
\{\pi_i \leftarrow P(\sigma, (x_i, y), w_i)\, i \in [1,q]\}
\end{array} 
\]

Since $g$ is PRF, it can be shown that $\mathcal{H}_0 \cong \mathcal{H}_1$ using Sunglass Lemma. Next, consider an intermediate hybrid $\mathcal{H}_{1.5}$ where everything is the same as $\mathcal{H}_1$, but the generation of $\pi_1$ is simulated by $\mathcal{S}_1, \mathcal{S}_2$. $\mathcal{H}_1$ and $\mathcal{H}_2$ are independently indistinguishable from $\mathcal{H}_{1.5}$, $\mathcal{H}_1 \cong \mathcal{H}_{1.5} \cong \mathcal{H}_2$. This process can be extended the argument to $\mathcal{H}_3, \mathcal{H}_4,...$ where the next $\pi$s are generated with $s$ in the same way. Therefore, by the hybrid argument, $\mathcal{H}_1$ is indistinguishable from a sequence where all proofs are generated with $s$. Thus, $\mathcal{S}_1, \mathcal{S}_2$ simulates the proofs by $P'$.



\subsection{Naor-Yung Construction}

The Naor-Yung construction relies on a semantically-secure public-key encryption scheme $(\mathsf{Gen}, \mathsf{Enc}, \mathsf{Dec})$ and an adaptively-secure NIZK proof system $(K, P, V)$ to construct a public-key encryption scheme $(\mathsf{Gen_{cca1}}, \mathsf{Enc_{cca1}}, \mathsf{Dec_{cca1}})$ that is CCA1 secure. The scheme is defined as follows:

\begin{itemize}
    \item $\mathsf{Gen_{cca1}}(1^n): $
    \begin{enumerate}
        \item $(pk_0, sk_0) \gets \mathsf{Gen}(1^n)$
        \item $(pk_1, sk_1) \gets \mathsf{Gen}(1^n)$
        \item $\sigma \gets K(1^n)$
        \item Output $pk = (pk_0, pk_1, \sigma)$, $sk = sk_0$
    \end{enumerate}
    \item $\mathsf{Enc_{cca1}}(pk = (pk_0, pk_1, \sigma), m): $
    \begin{enumerate}
        \item $\{c_b \gets \mathsf{Enc}(pk_b, m; r_b)\}_{b\in\bit}$
        \item $\Pi \gets P(\sigma, (c_0, c_1), (m, r_0, r_1))$
        \item Output $(c_0, c_1, \Pi)$
    \end{enumerate}
    \item  $\mathsf{Dec_{cca1}}(sk = sk_0, CT = (c_0, c_1, \Pi)): $
    \begin{enumerate}
        \item If $V(\sigma, (c_0, c_1), \Pi) = 1$: output $\mathsf{Dec}(sk_0, c_0)$
        \item Else: output $\bot$
    \end{enumerate}
\end{itemize}

\begin{theorem}\label{thm:naor-yung-cca1}
    Assuming $(\mathsf{Gen}, \mathsf{Enc}, \mathsf{Dec})$ is a semantically-secure encryption scheme and $(K, P, V)$ is an adaptively-secure NIZK proof system, then $(\mathsf{Gen_{cca1}}, \mathsf{Enc_{cca1}}, \mathsf{Dec_{cca1}})$ is secure against non-adaptive chosen-ciphertext attacks. 
\end{theorem}
\proof 
Recall the definition of CCA1 security game:

\begin{center}\begin{tabular}{ r c l }
  \textbf{Challenger} & & \textbf{Adversary} \\
  $(pk, sk) \gets \mathsf{Gen}(1^n)$ & $\xrightarrow{pk = (pk_0, pk_1, \sigma)}$ &  \\
   & $\xleftarrow{CT_1}$ & \\
  $m_1 \gets \mathsf{Dec}(sk, CT_1)$ & $\xrightarrow{m_1}$ & \\
  $\vdots$ & $\vdots$ & $\vdots$ \\
   & $\xleftarrow{CT_i}$ & \\
  $m_i \gets \mathsf{Dec}(sk, CT_i)$ & $\xrightarrow{m_i}$ & \\
   & $\xleftarrow{m_0, m_1}$ & \\
  $b \overset{\$}{\gets} \{0, 1\}$, $c^* = \mathsf{Enc}(pk, m_b)$ & $\xrightarrow{c^* = (c_0^*, c_1^*, \Pi^*)}$ & \\
  Output $1$ if $b' = b$, otherwise $0$ & $\xleftarrow{b'}$ & \\
\end{tabular}
\end{center}

Starting from this Game 0, we can derive Game 1 to 4, so that it is clear that at Game 4, the adversary does not receive any information about $b$ from $c^*$, and hence it is impossible to correctly guess $b$ with a non-negligible advantage. The intuition here is to replace $c_0^*, c_1^*, \Pi^*$ step by step. An important lemma that we make use of in the construction is:
\begin{lemma} \label{lemma:naor-yung}
For $\forall CT = (c_0, c_1, \Pi)$, we have that if $V(\sigma, (c_0, c_1), \Pi) = 1$, $\mathsf{Dec}(sk_0, c_0) = \mathsf{Dec}(sk_1, c_1).$
\end{lemma}

The lemma is a direct result of the soundness guarantee of the NICK proof system. Now we are ready to construct the games:
\begin{itemize}
    \item Game 1: instead of $(P, V)$, its simulator is used to provide the proof $\Pi^*$. Specifically, we  have $(\sigma, \tau) \gets S_1(1^n)$ and $\Pi^* \gets S_2(\tau, (c_0^*, c_1^*))$. Now $\Pi^*$ does not contain information about $b$.
    \item Game 2: note that only $sk_0$ is used to answer all the decryption queries by the adversary, as a result, we can set $c_1^*\gets \mathsf{Enc}(pk_1, 0^{|m_1|})$. This works because of the indistinguishability of the public-key encryption scheme. Now $c_1^*$ does not contain information about $b$.
    \item Game 3: using Lemma~\ref{lemma:naor-yung}, we can use $sk_1$ instead of $sk_0$ to answer all the decryption queries by the adversary.
    \item Game 4: now, since only $sk_1$ is used to answer all the decryption queries by the adversary. we can set $c_0^*\gets \mathsf{Enc}(pk_0, 0^{|m_0|})$. This works because of the indistinguishability of the public-key encryption scheme. Now $c_0^*$ does not contain information about $b$ either.
\end{itemize}
